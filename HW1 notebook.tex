
% Default to the notebook output style

    


% Inherit from the specified cell style.




    
\documentclass[11pt]{article}

    
    
    \usepackage[T1]{fontenc}
    % Nicer default font (+ math font) than Computer Modern for most use cases
    \usepackage{mathpazo}

    % Basic figure setup, for now with no caption control since it's done
    % automatically by Pandoc (which extracts ![](path) syntax from Markdown).
    \usepackage{graphicx}
    % We will generate all images so they have a width \maxwidth. This means
    % that they will get their normal width if they fit onto the page, but
    % are scaled down if they would overflow the margins.
    \makeatletter
    \def\maxwidth{\ifdim\Gin@nat@width>\linewidth\linewidth
    \else\Gin@nat@width\fi}
    \makeatother
    \let\Oldincludegraphics\includegraphics
    % Set max figure width to be 80% of text width, for now hardcoded.
    \renewcommand{\includegraphics}[1]{\Oldincludegraphics[width=.8\maxwidth]{#1}}
    % Ensure that by default, figures have no caption (until we provide a
    % proper Figure object with a Caption API and a way to capture that
    % in the conversion process - todo).
    \usepackage{caption}
    \DeclareCaptionLabelFormat{nolabel}{}
    \captionsetup{labelformat=nolabel}

    \usepackage{adjustbox} % Used to constrain images to a maximum size 
    \usepackage{xcolor} % Allow colors to be defined
    \usepackage{enumerate} % Needed for markdown enumerations to work
    \usepackage{geometry} % Used to adjust the document margins
    \usepackage{amsmath} % Equations
    \usepackage{amssymb} % Equations
    \usepackage{textcomp} % defines textquotesingle
    % Hack from http://tex.stackexchange.com/a/47451/13684:
    \AtBeginDocument{%
        \def\PYZsq{\textquotesingle}% Upright quotes in Pygmentized code
    }
    \usepackage{upquote} % Upright quotes for verbatim code
    \usepackage{eurosym} % defines \euro
    \usepackage[mathletters]{ucs} % Extended unicode (utf-8) support
    \usepackage[utf8x]{inputenc} % Allow utf-8 characters in the tex document
    \usepackage{fancyvrb} % verbatim replacement that allows latex
    \usepackage{grffile} % extends the file name processing of package graphics 
                         % to support a larger range 
    % The hyperref package gives us a pdf with properly built
    % internal navigation ('pdf bookmarks' for the table of contents,
    % internal cross-reference links, web links for URLs, etc.)
    \usepackage{hyperref}
    \usepackage{longtable} % longtable support required by pandoc >1.10
    \usepackage{booktabs}  % table support for pandoc > 1.12.2
    \usepackage[inline]{enumitem} % IRkernel/repr support (it uses the enumerate* environment)
    \usepackage[normalem]{ulem} % ulem is needed to support strikethroughs (\sout)
                                % normalem makes italics be italics, not underlines
    \usepackage{mathrsfs}
    

    
    
    % Colors for the hyperref package
    \definecolor{urlcolor}{rgb}{0,.145,.698}
    \definecolor{linkcolor}{rgb}{.71,0.21,0.01}
    \definecolor{citecolor}{rgb}{.12,.54,.11}

    % ANSI colors
    \definecolor{ansi-black}{HTML}{3E424D}
    \definecolor{ansi-black-intense}{HTML}{282C36}
    \definecolor{ansi-red}{HTML}{E75C58}
    \definecolor{ansi-red-intense}{HTML}{B22B31}
    \definecolor{ansi-green}{HTML}{00A250}
    \definecolor{ansi-green-intense}{HTML}{007427}
    \definecolor{ansi-yellow}{HTML}{DDB62B}
    \definecolor{ansi-yellow-intense}{HTML}{B27D12}
    \definecolor{ansi-blue}{HTML}{208FFB}
    \definecolor{ansi-blue-intense}{HTML}{0065CA}
    \definecolor{ansi-magenta}{HTML}{D160C4}
    \definecolor{ansi-magenta-intense}{HTML}{A03196}
    \definecolor{ansi-cyan}{HTML}{60C6C8}
    \definecolor{ansi-cyan-intense}{HTML}{258F8F}
    \definecolor{ansi-white}{HTML}{C5C1B4}
    \definecolor{ansi-white-intense}{HTML}{A1A6B2}
    \definecolor{ansi-default-inverse-fg}{HTML}{FFFFFF}
    \definecolor{ansi-default-inverse-bg}{HTML}{000000}

    % commands and environments needed by pandoc snippets
    % extracted from the output of `pandoc -s`
    \providecommand{\tightlist}{%
      \setlength{\itemsep}{0pt}\setlength{\parskip}{0pt}}
    \DefineVerbatimEnvironment{Highlighting}{Verbatim}{commandchars=\\\{\}}
    % Add ',fontsize=\small' for more characters per line
    \newenvironment{Shaded}{}{}
    \newcommand{\KeywordTok}[1]{\textcolor[rgb]{0.00,0.44,0.13}{\textbf{{#1}}}}
    \newcommand{\DataTypeTok}[1]{\textcolor[rgb]{0.56,0.13,0.00}{{#1}}}
    \newcommand{\DecValTok}[1]{\textcolor[rgb]{0.25,0.63,0.44}{{#1}}}
    \newcommand{\BaseNTok}[1]{\textcolor[rgb]{0.25,0.63,0.44}{{#1}}}
    \newcommand{\FloatTok}[1]{\textcolor[rgb]{0.25,0.63,0.44}{{#1}}}
    \newcommand{\CharTok}[1]{\textcolor[rgb]{0.25,0.44,0.63}{{#1}}}
    \newcommand{\StringTok}[1]{\textcolor[rgb]{0.25,0.44,0.63}{{#1}}}
    \newcommand{\CommentTok}[1]{\textcolor[rgb]{0.38,0.63,0.69}{\textit{{#1}}}}
    \newcommand{\OtherTok}[1]{\textcolor[rgb]{0.00,0.44,0.13}{{#1}}}
    \newcommand{\AlertTok}[1]{\textcolor[rgb]{1.00,0.00,0.00}{\textbf{{#1}}}}
    \newcommand{\FunctionTok}[1]{\textcolor[rgb]{0.02,0.16,0.49}{{#1}}}
    \newcommand{\RegionMarkerTok}[1]{{#1}}
    \newcommand{\ErrorTok}[1]{\textcolor[rgb]{1.00,0.00,0.00}{\textbf{{#1}}}}
    \newcommand{\NormalTok}[1]{{#1}}
    
    % Additional commands for more recent versions of Pandoc
    \newcommand{\ConstantTok}[1]{\textcolor[rgb]{0.53,0.00,0.00}{{#1}}}
    \newcommand{\SpecialCharTok}[1]{\textcolor[rgb]{0.25,0.44,0.63}{{#1}}}
    \newcommand{\VerbatimStringTok}[1]{\textcolor[rgb]{0.25,0.44,0.63}{{#1}}}
    \newcommand{\SpecialStringTok}[1]{\textcolor[rgb]{0.73,0.40,0.53}{{#1}}}
    \newcommand{\ImportTok}[1]{{#1}}
    \newcommand{\DocumentationTok}[1]{\textcolor[rgb]{0.73,0.13,0.13}{\textit{{#1}}}}
    \newcommand{\AnnotationTok}[1]{\textcolor[rgb]{0.38,0.63,0.69}{\textbf{\textit{{#1}}}}}
    \newcommand{\CommentVarTok}[1]{\textcolor[rgb]{0.38,0.63,0.69}{\textbf{\textit{{#1}}}}}
    \newcommand{\VariableTok}[1]{\textcolor[rgb]{0.10,0.09,0.49}{{#1}}}
    \newcommand{\ControlFlowTok}[1]{\textcolor[rgb]{0.00,0.44,0.13}{\textbf{{#1}}}}
    \newcommand{\OperatorTok}[1]{\textcolor[rgb]{0.40,0.40,0.40}{{#1}}}
    \newcommand{\BuiltInTok}[1]{{#1}}
    \newcommand{\ExtensionTok}[1]{{#1}}
    \newcommand{\PreprocessorTok}[1]{\textcolor[rgb]{0.74,0.48,0.00}{{#1}}}
    \newcommand{\AttributeTok}[1]{\textcolor[rgb]{0.49,0.56,0.16}{{#1}}}
    \newcommand{\InformationTok}[1]{\textcolor[rgb]{0.38,0.63,0.69}{\textbf{\textit{{#1}}}}}
    \newcommand{\WarningTok}[1]{\textcolor[rgb]{0.38,0.63,0.69}{\textbf{\textit{{#1}}}}}
    
    
    % Define a nice break command that doesn't care if a line doesn't already
    % exist.
    \def\br{\hspace*{\fill} \\* }
    % Math Jax compatibility definitions
    \def\gt{>}
    \def\lt{<}
    \let\Oldtex\TeX
    \let\Oldlatex\LaTeX
    \renewcommand{\TeX}{\textrm{\Oldtex}}
    \renewcommand{\LaTeX}{\textrm{\Oldlatex}}
    % Document parameters
    % Document title
    \title{HW1 notebook}
    
    
    
    
    

    % Pygments definitions
    
\makeatletter
\def\PY@reset{\let\PY@it=\relax \let\PY@bf=\relax%
    \let\PY@ul=\relax \let\PY@tc=\relax%
    \let\PY@bc=\relax \let\PY@ff=\relax}
\def\PY@tok#1{\csname PY@tok@#1\endcsname}
\def\PY@toks#1+{\ifx\relax#1\empty\else%
    \PY@tok{#1}\expandafter\PY@toks\fi}
\def\PY@do#1{\PY@bc{\PY@tc{\PY@ul{%
    \PY@it{\PY@bf{\PY@ff{#1}}}}}}}
\def\PY#1#2{\PY@reset\PY@toks#1+\relax+\PY@do{#2}}

\expandafter\def\csname PY@tok@w\endcsname{\def\PY@tc##1{\textcolor[rgb]{0.73,0.73,0.73}{##1}}}
\expandafter\def\csname PY@tok@c\endcsname{\let\PY@it=\textit\def\PY@tc##1{\textcolor[rgb]{0.25,0.50,0.50}{##1}}}
\expandafter\def\csname PY@tok@cp\endcsname{\def\PY@tc##1{\textcolor[rgb]{0.74,0.48,0.00}{##1}}}
\expandafter\def\csname PY@tok@k\endcsname{\let\PY@bf=\textbf\def\PY@tc##1{\textcolor[rgb]{0.00,0.50,0.00}{##1}}}
\expandafter\def\csname PY@tok@kp\endcsname{\def\PY@tc##1{\textcolor[rgb]{0.00,0.50,0.00}{##1}}}
\expandafter\def\csname PY@tok@kt\endcsname{\def\PY@tc##1{\textcolor[rgb]{0.69,0.00,0.25}{##1}}}
\expandafter\def\csname PY@tok@o\endcsname{\def\PY@tc##1{\textcolor[rgb]{0.40,0.40,0.40}{##1}}}
\expandafter\def\csname PY@tok@ow\endcsname{\let\PY@bf=\textbf\def\PY@tc##1{\textcolor[rgb]{0.67,0.13,1.00}{##1}}}
\expandafter\def\csname PY@tok@nb\endcsname{\def\PY@tc##1{\textcolor[rgb]{0.00,0.50,0.00}{##1}}}
\expandafter\def\csname PY@tok@nf\endcsname{\def\PY@tc##1{\textcolor[rgb]{0.00,0.00,1.00}{##1}}}
\expandafter\def\csname PY@tok@nc\endcsname{\let\PY@bf=\textbf\def\PY@tc##1{\textcolor[rgb]{0.00,0.00,1.00}{##1}}}
\expandafter\def\csname PY@tok@nn\endcsname{\let\PY@bf=\textbf\def\PY@tc##1{\textcolor[rgb]{0.00,0.00,1.00}{##1}}}
\expandafter\def\csname PY@tok@ne\endcsname{\let\PY@bf=\textbf\def\PY@tc##1{\textcolor[rgb]{0.82,0.25,0.23}{##1}}}
\expandafter\def\csname PY@tok@nv\endcsname{\def\PY@tc##1{\textcolor[rgb]{0.10,0.09,0.49}{##1}}}
\expandafter\def\csname PY@tok@no\endcsname{\def\PY@tc##1{\textcolor[rgb]{0.53,0.00,0.00}{##1}}}
\expandafter\def\csname PY@tok@nl\endcsname{\def\PY@tc##1{\textcolor[rgb]{0.63,0.63,0.00}{##1}}}
\expandafter\def\csname PY@tok@ni\endcsname{\let\PY@bf=\textbf\def\PY@tc##1{\textcolor[rgb]{0.60,0.60,0.60}{##1}}}
\expandafter\def\csname PY@tok@na\endcsname{\def\PY@tc##1{\textcolor[rgb]{0.49,0.56,0.16}{##1}}}
\expandafter\def\csname PY@tok@nt\endcsname{\let\PY@bf=\textbf\def\PY@tc##1{\textcolor[rgb]{0.00,0.50,0.00}{##1}}}
\expandafter\def\csname PY@tok@nd\endcsname{\def\PY@tc##1{\textcolor[rgb]{0.67,0.13,1.00}{##1}}}
\expandafter\def\csname PY@tok@s\endcsname{\def\PY@tc##1{\textcolor[rgb]{0.73,0.13,0.13}{##1}}}
\expandafter\def\csname PY@tok@sd\endcsname{\let\PY@it=\textit\def\PY@tc##1{\textcolor[rgb]{0.73,0.13,0.13}{##1}}}
\expandafter\def\csname PY@tok@si\endcsname{\let\PY@bf=\textbf\def\PY@tc##1{\textcolor[rgb]{0.73,0.40,0.53}{##1}}}
\expandafter\def\csname PY@tok@se\endcsname{\let\PY@bf=\textbf\def\PY@tc##1{\textcolor[rgb]{0.73,0.40,0.13}{##1}}}
\expandafter\def\csname PY@tok@sr\endcsname{\def\PY@tc##1{\textcolor[rgb]{0.73,0.40,0.53}{##1}}}
\expandafter\def\csname PY@tok@ss\endcsname{\def\PY@tc##1{\textcolor[rgb]{0.10,0.09,0.49}{##1}}}
\expandafter\def\csname PY@tok@sx\endcsname{\def\PY@tc##1{\textcolor[rgb]{0.00,0.50,0.00}{##1}}}
\expandafter\def\csname PY@tok@m\endcsname{\def\PY@tc##1{\textcolor[rgb]{0.40,0.40,0.40}{##1}}}
\expandafter\def\csname PY@tok@gh\endcsname{\let\PY@bf=\textbf\def\PY@tc##1{\textcolor[rgb]{0.00,0.00,0.50}{##1}}}
\expandafter\def\csname PY@tok@gu\endcsname{\let\PY@bf=\textbf\def\PY@tc##1{\textcolor[rgb]{0.50,0.00,0.50}{##1}}}
\expandafter\def\csname PY@tok@gd\endcsname{\def\PY@tc##1{\textcolor[rgb]{0.63,0.00,0.00}{##1}}}
\expandafter\def\csname PY@tok@gi\endcsname{\def\PY@tc##1{\textcolor[rgb]{0.00,0.63,0.00}{##1}}}
\expandafter\def\csname PY@tok@gr\endcsname{\def\PY@tc##1{\textcolor[rgb]{1.00,0.00,0.00}{##1}}}
\expandafter\def\csname PY@tok@ge\endcsname{\let\PY@it=\textit}
\expandafter\def\csname PY@tok@gs\endcsname{\let\PY@bf=\textbf}
\expandafter\def\csname PY@tok@gp\endcsname{\let\PY@bf=\textbf\def\PY@tc##1{\textcolor[rgb]{0.00,0.00,0.50}{##1}}}
\expandafter\def\csname PY@tok@go\endcsname{\def\PY@tc##1{\textcolor[rgb]{0.53,0.53,0.53}{##1}}}
\expandafter\def\csname PY@tok@gt\endcsname{\def\PY@tc##1{\textcolor[rgb]{0.00,0.27,0.87}{##1}}}
\expandafter\def\csname PY@tok@err\endcsname{\def\PY@bc##1{\setlength{\fboxsep}{0pt}\fcolorbox[rgb]{1.00,0.00,0.00}{1,1,1}{\strut ##1}}}
\expandafter\def\csname PY@tok@kc\endcsname{\let\PY@bf=\textbf\def\PY@tc##1{\textcolor[rgb]{0.00,0.50,0.00}{##1}}}
\expandafter\def\csname PY@tok@kd\endcsname{\let\PY@bf=\textbf\def\PY@tc##1{\textcolor[rgb]{0.00,0.50,0.00}{##1}}}
\expandafter\def\csname PY@tok@kn\endcsname{\let\PY@bf=\textbf\def\PY@tc##1{\textcolor[rgb]{0.00,0.50,0.00}{##1}}}
\expandafter\def\csname PY@tok@kr\endcsname{\let\PY@bf=\textbf\def\PY@tc##1{\textcolor[rgb]{0.00,0.50,0.00}{##1}}}
\expandafter\def\csname PY@tok@bp\endcsname{\def\PY@tc##1{\textcolor[rgb]{0.00,0.50,0.00}{##1}}}
\expandafter\def\csname PY@tok@fm\endcsname{\def\PY@tc##1{\textcolor[rgb]{0.00,0.00,1.00}{##1}}}
\expandafter\def\csname PY@tok@vc\endcsname{\def\PY@tc##1{\textcolor[rgb]{0.10,0.09,0.49}{##1}}}
\expandafter\def\csname PY@tok@vg\endcsname{\def\PY@tc##1{\textcolor[rgb]{0.10,0.09,0.49}{##1}}}
\expandafter\def\csname PY@tok@vi\endcsname{\def\PY@tc##1{\textcolor[rgb]{0.10,0.09,0.49}{##1}}}
\expandafter\def\csname PY@tok@vm\endcsname{\def\PY@tc##1{\textcolor[rgb]{0.10,0.09,0.49}{##1}}}
\expandafter\def\csname PY@tok@sa\endcsname{\def\PY@tc##1{\textcolor[rgb]{0.73,0.13,0.13}{##1}}}
\expandafter\def\csname PY@tok@sb\endcsname{\def\PY@tc##1{\textcolor[rgb]{0.73,0.13,0.13}{##1}}}
\expandafter\def\csname PY@tok@sc\endcsname{\def\PY@tc##1{\textcolor[rgb]{0.73,0.13,0.13}{##1}}}
\expandafter\def\csname PY@tok@dl\endcsname{\def\PY@tc##1{\textcolor[rgb]{0.73,0.13,0.13}{##1}}}
\expandafter\def\csname PY@tok@s2\endcsname{\def\PY@tc##1{\textcolor[rgb]{0.73,0.13,0.13}{##1}}}
\expandafter\def\csname PY@tok@sh\endcsname{\def\PY@tc##1{\textcolor[rgb]{0.73,0.13,0.13}{##1}}}
\expandafter\def\csname PY@tok@s1\endcsname{\def\PY@tc##1{\textcolor[rgb]{0.73,0.13,0.13}{##1}}}
\expandafter\def\csname PY@tok@mb\endcsname{\def\PY@tc##1{\textcolor[rgb]{0.40,0.40,0.40}{##1}}}
\expandafter\def\csname PY@tok@mf\endcsname{\def\PY@tc##1{\textcolor[rgb]{0.40,0.40,0.40}{##1}}}
\expandafter\def\csname PY@tok@mh\endcsname{\def\PY@tc##1{\textcolor[rgb]{0.40,0.40,0.40}{##1}}}
\expandafter\def\csname PY@tok@mi\endcsname{\def\PY@tc##1{\textcolor[rgb]{0.40,0.40,0.40}{##1}}}
\expandafter\def\csname PY@tok@il\endcsname{\def\PY@tc##1{\textcolor[rgb]{0.40,0.40,0.40}{##1}}}
\expandafter\def\csname PY@tok@mo\endcsname{\def\PY@tc##1{\textcolor[rgb]{0.40,0.40,0.40}{##1}}}
\expandafter\def\csname PY@tok@ch\endcsname{\let\PY@it=\textit\def\PY@tc##1{\textcolor[rgb]{0.25,0.50,0.50}{##1}}}
\expandafter\def\csname PY@tok@cm\endcsname{\let\PY@it=\textit\def\PY@tc##1{\textcolor[rgb]{0.25,0.50,0.50}{##1}}}
\expandafter\def\csname PY@tok@cpf\endcsname{\let\PY@it=\textit\def\PY@tc##1{\textcolor[rgb]{0.25,0.50,0.50}{##1}}}
\expandafter\def\csname PY@tok@c1\endcsname{\let\PY@it=\textit\def\PY@tc##1{\textcolor[rgb]{0.25,0.50,0.50}{##1}}}
\expandafter\def\csname PY@tok@cs\endcsname{\let\PY@it=\textit\def\PY@tc##1{\textcolor[rgb]{0.25,0.50,0.50}{##1}}}

\def\PYZbs{\char`\\}
\def\PYZus{\char`\_}
\def\PYZob{\char`\{}
\def\PYZcb{\char`\}}
\def\PYZca{\char`\^}
\def\PYZam{\char`\&}
\def\PYZlt{\char`\<}
\def\PYZgt{\char`\>}
\def\PYZsh{\char`\#}
\def\PYZpc{\char`\%}
\def\PYZdl{\char`\$}
\def\PYZhy{\char`\-}
\def\PYZsq{\char`\'}
\def\PYZdq{\char`\"}
\def\PYZti{\char`\~}
% for compatibility with earlier versions
\def\PYZat{@}
\def\PYZlb{[}
\def\PYZrb{]}
\makeatother


    % Exact colors from NB
    \definecolor{incolor}{rgb}{0.0, 0.0, 0.5}
    \definecolor{outcolor}{rgb}{0.545, 0.0, 0.0}



    
    % Prevent overflowing lines due to hard-to-break entities
    \sloppy 
    % Setup hyperref package
    \hypersetup{
      breaklinks=true,  % so long urls are correctly broken across lines
      colorlinks=true,
      urlcolor=urlcolor,
      linkcolor=linkcolor,
      citecolor=citecolor,
      }
    % Slightly bigger margins than the latex defaults
    
    \geometry{verbose,tmargin=1in,bmargin=1in,lmargin=1in,rmargin=1in}
    
    

    \begin{document}
    
    
    \maketitle
    
    

    
    Introduction to Artificial Intelligence 236501 Homework 1

Yoni Ben-Zvi 203668900 Danny Priymak 307003434

    \subsubsection{Question 1}\label{question-1}

The requested table is produced by the following code:

    \begin{Verbatim}[commandchars=\\\{\}]
{\color{incolor}In [{\color{incolor}1}]:} \PY{k+kn}{from} \PY{n+nn}{tabulate} \PY{k}{import} \PY{n}{tabulate}
        \PY{k+kn}{import} \PY{n+nn}{math}
        
        \PY{n}{list1} \PY{o}{=} \PY{p}{[}\PY{p}{]}
        \PY{k}{for} \PY{n}{x} \PY{o+ow}{in} \PY{n+nb}{range}\PY{p}{(}\PY{l+m+mi}{1}\PY{p}{,} \PY{l+m+mi}{11}\PY{p}{)}\PY{p}{:}
            \PY{n}{list2} \PY{o}{=} \PY{p}{[}\PY{n}{x}\PY{p}{,} \PY{n}{math}\PY{o}{.}\PY{n}{factorial}\PY{p}{(}\PY{n}{x}\PY{p}{)}\PY{p}{,} \PY{n}{math}\PY{o}{.}\PY{n}{factorial}\PY{p}{(}\PY{n}{x}\PY{p}{)} \PY{o}{*} \PY{p}{(}\PY{l+m+mi}{5} \PY{o}{*}\PY{o}{*} \PY{p}{(}\PY{n}{x} \PY{o}{\PYZhy{}} \PY{l+m+mi}{1}\PY{p}{)}\PY{p}{)}\PY{p}{]}
            \PY{n}{list1}\PY{o}{.}\PY{n}{append}\PY{p}{(}\PY{n}{list2}\PY{p}{)}
        \PY{n+nb}{print}\PY{p}{(}\PY{n}{tabulate}\PY{p}{(}\PY{n}{list1}\PY{p}{,} \PY{n}{headers}\PY{o}{=}\PY{p}{[}\PY{l+s+s1}{\PYZsq{}}\PY{l+s+s1}{k}\PY{l+s+s1}{\PYZsq{}}\PY{p}{,} \PY{l+s+s1}{\PYZsq{}}\PY{l+s+s1}{l = 0}\PY{l+s+s1}{\PYZsq{}}\PY{p}{,} \PY{l+s+s1}{\PYZsq{}}\PY{l+s+s1}{l = 5}\PY{l+s+s1}{\PYZsq{}}\PY{p}{]}\PY{p}{,} \PY{n}{tablefmt}\PY{o}{=}\PY{l+s+s1}{\PYZsq{}}\PY{l+s+s1}{orgtbl}\PY{l+s+s1}{\PYZsq{}}\PY{p}{)}\PY{p}{)}
\end{Verbatim}

    \begin{Verbatim}[commandchars=\\\{\}]
|   k |   l = 0 |         l = 5 |
|-----+---------+---------------|
|   1 |       1 |             1 |
|   2 |       2 |            10 |
|   3 |       6 |           150 |
|   4 |      24 |          3000 |
|   5 |     120 |         75000 |
|   6 |     720 |       2250000 |
|   7 |    5040 |      78750000 |
|   8 |   40320 |    3150000000 |
|   9 |  362880 |  141750000000 |
|  10 | 3628800 | 7087500000000 |

    \end{Verbatim}

    \subsubsection{Question 2}\label{question-2}

The branching factor's extremum values are: Maximum = \(k+l\) because if
our graph is a complete graph, each vertex is connected to all other gas
stations and all other delivery points. If the current vertex is
\(v_0\), then it is connected to all other \(k+l\) vertices of the
graph. This is the maximal number of vertices it can be connected to
using the operators defined in the exercise. Minimum = \(1\) because if
the graph is 1-regular then the maximal outdegree of each vertex
is 1, and therefore the graph's branching factor is 1.

    \subsubsection{Question 3}\label{question-3}

Yes. Consider a graph that contains two gas station vertices \(f_1,f_2\)
that are connected to each other, and the distance between them is less
than the maximal distance the scooter can go when its fuel tank is full.
Suppose the scooter is currently at \(f_1\). Therefore, its fuel tank is
full. Let us denote this state as \(S_1\). Suppose it chooses to go to
\(f_2\). Once it arrives at \(f_2\), its fuel level is once again full.
Lastly, suppose the scooter chooses to go back to \(f_1\). Once it
arrives there (for the second time), its fuel capacity is once again
full. Note that the current state is identical to \(S_1\), as The sets
\(T,F\) have not changed during this path's traversal, and the fuel
levels were constant. Therefore, we have found a directed cycle in the
graph.

    \subsubsection{Question 4}\label{question-4}

Let us first distinguish between the start state
\(\left(v_0,d_0,\textrm{Ord},\emptyset \right)\) and all other states.
Since we assume no order or gas station can be placed in \(v_0\), the
only time the scooter will be at \(v_0\) is when the algorithm starts.
Hence, we can count this state only once.

Once we have the distinction above, the number of options for the
scooter's location is \(k+l\), as it can only stop at gas stations
(which there are \(l\) of) or delivery drop-off locations (which there
are \(k\) of).

As for the fuel level \(d\), if we assume the possible fuel levels are
real values between \(0\) and \(d_\textrm{refuel}\) in, say,
double-precision, then the number of options is
\(2^{64}=18446744073709551616\) (with maximal resolution possible with
respect to the real value range \([0,d_\textrm{refuel}]\)).

As was pointed out in the exercise, we can keep track of only one of the
sets \(T,F\), since the other is its complimentary set with respect to
\([k] =\{ 1,\dots,k\}\). Without loss of generality let us keep track of
only \(T\). Hence, the number of possible sets \(T\) is
\(|\mathcal{P}([k])|=2^k\), where \(\mathcal{P}()\) denotes the power
set.

To summarize, the resulting number of states is given by the formula
\[((k+l)\cdot (2^{64})\cdot 2^{k})+1\]

    \subsubsection{Question 5}\label{question-5}

Yes. Assume that \(|T|\geq 2\) and that the scooter has just enough gas
to reach a certain delivery drop-off point \(t_i\) from its current
location, and have exactly zero fuel left upon arrival at \(t_i\). The
operation \(|T|\leftarrow |T|-1\) is computed, and the scooter cannot go
further to any delivery drop-off point nor gas station in the graph, and
the new state it's at is not a goal state since \(|T|\geq 1\).

    \subsubsection{Question 6}\label{question-6}

\[\text{Succ}_{1}\left(\left(v_{1},d_{1},T_{1},F_{1}\right)\right) = \left\{ \left(v_{2},d_{2},T_{2},F_{2}\right)\in S:\ \substack{v_{2}\in\text{ Ord }\\
d_{2}=d_{1}-\text{ Dist}\left(v_{1},v_{2}\right)\quad\land\quad d_{1}-\text{ Dist}\left(v_{1},v_{2}\right)\geq0\\
\exists i\in\left[k\right]:\quad i\in T_{1}\quad\land\quad T_{2}=T_{1}\setminus\left\{ i\right\} \quad\land\quad F_{2}=F_{1}\cup\left\{ i\right\} \\
\text{There exists a directed path \ensuremath{v_{1}\to\dots\to v_{2}} on the map}
}
\right\}\] \[
\text{Succ}_{2}\left(\left(v_{1},d_{1},T_{1},F_{1}\right)\right)    =\left\{ \left(v_{2},d_{2},T_{2},F_{2}\right)\in S:\ \substack{v_{2}\in\text{ GasStations }\\
d_{2}=d_{\text{refuel}}\quad\land\quad d_{1}-\text{ Dist}\left(v_{1},v_{2}\right)\geq0\\
T_{1}=T_{2}\quad\land\quad F_{1}=F_{2}\\
\text{There exists a directed path \ensuremath{v_{1}\to\dots\to v_{2}} on the map}
}
\right\} \] \[
\text{Succ}{\left(\left(v_{1},d_{1},T_{1},F_{1}\right)\right)}  =\text{Succ}_{1}\left(\left(v_{1},d_{1},T_{1},F_{1}\right)\right)\ \cup\ \text{Succ}_{2}\left(\left(v_{1},d_{1},T_{1},F_{1}\right)\right)\]

    \subsubsection{Question 7}\label{question-7}

If we ignore the fuel constraint and assume that \(d_0\) is very big
such that the scooter does not need to refuel during its trip, the goal
state minimal depth must be at least the number of delivery drop-off
points \(k\), as the scooter must go through all of them to get to a
goal state. If we reconsider the fuel constraint, each refuel operation
adds one level of depth to the search, hence increasing the depth.
Therefore, the minimal depth is \(k\).

    \subsubsection{Question 8}\label{question-8}

load\_map\_from\_csv: 1.52sec

Solve the map problem. Map(src: 54 dst: 549) UniformCost time: 0.59
\#dev: 17355 total\_cost: 7465.52897 \textbar{}path\textbar{}: 137 path:
{[} 54, 55, 56, 57, 58, 59, 60, 28893, 14580, 14590, 14591, 14592,
14593, 81892, 25814, 81, 26236, 26234, 1188, 33068, 33069, 33070, 15474,
33071, 5020, 21699, 33072, 33073, 33074, 16203, 9847, 9848, 9849, 9850,
9851, 335, 9852, 82906, 82907, 82908, 82909, 95454, 96539, 72369, 94627,
38553, 72367, 29007, 94632, 96540, 9269, 82890, 29049, 29026, 82682,
71897, 83380, 96541, 82904, 96542, 96543, 96544, 96545, 96546, 96547,
82911, 82928, 24841, 24842, 24843, 5215, 24844, 9274, 24845, 24846,
24847, 24848, 24849, 24850, 24851, 24852, 24853, 24854, 24855, 24856,
24857, 24858, 24859, 24860, 24861, 24862, 24863, 24864, 24865, 24866,
82208, 82209, 82210, 21518, 21431, 21432, 21433, 21434, 21435, 21436,
21437, 21438, 21439, 21440, 21441, 21442, 21443, 21444, 21445, 21446,
21447, 21448, 21449, 21450, 21451, 621, 21452, 21453, 21454, 21495,
21496, 539, 540, 541, 542, 543, 544, 545, 546, 547, 548, 549{]}

    \subsubsection{Question 11}\label{question-11}

load\_map\_from\_csv: 1.39sec

Solve the map problem. Map(src: 54 dst: 549) A* (h=AirDist, w=0.500)
time: 0.09 \#dev: 2016 total\_cost: 7465.52897 \textbar{}path\textbar{}:
137 path: {[} 54, 55, 56, 57, 58, 59, 60, 28893, 14580, 14590, 14591,
14592, 14593, 81892, 25814, 81, 26236, 26234, 1188, 33068, 33069, 33070,
15474, 33071, 5020, 21699, 33072, 33073, 33074, 16203, 9847, 9848, 9849,
9850, 9851, 335, 9852, 82906, 82907, 82908, 82909, 95454, 96539, 72369,
94627, 38553, 72367, 29007, 94632, 96540, 9269, 82890, 29049, 29026,
82682, 71897, 83380, 96541, 82904, 96542, 96543, 96544, 96545, 96546,
96547, 82911, 82928, 24841, 24842, 24843, 5215, 24844, 9274, 24845,
24846, 24847, 24848, 24849, 24850, 24851, 24852, 24853, 24854, 24855,
24856, 24857, 24858, 24859, 24860, 24861, 24862, 24863, 24864, 24865,
24866, 82208, 82209, 82210, 21518, 21431, 21432, 21433, 21434, 21435,
21436, 21437, 21438, 21439, 21440, 21441, 21442, 21443, 21444, 21445,
21446, 21447, 21448, 21449, 21450, 21451, 621, 21452, 21453, 21454,
21495, 21496, 539, 540, 541, 542, 543, 544, 545, 546, 547, 548, 549{]}

    \subsubsection{Question 12}\label{question-12}

When the weight is 0.5, the algorithm is exactly the A* algorithm. As
the weight increases, the algorithm gets closer to the greedy best
search algorithm, with a weight value of 1 being the greedy best search
algorithm itself.

As the weight increases, the algorithm relies more and more on the
heuristic function and less and less on the cost of the current path, as
we've seen in class. In addition, as the weight increases, we can see
that the computation becomes computationally easier as less nodes are
being expanded. 

    \subsubsection{Question 14}\label{question-14}

The MaxAirDist heuristic is indeed admissable, since it does not take
into account all other nodes that need to be visited excecpt for the
node with the maximal air distance from the current node that has not
been visited yet.

In other words, if there is only one more node that needs to be visited
before a goal state is reached, the MaxAirDist heuristic will return the
exact distance to the goal state, which is \(h^{*}\), else, it will
return a smaller value than \(h^{*}\), hence it is indeed admissable.

    \subsubsection{Question 16}\label{question-16}

Solve the relaxed deliveries problem. RelaxedDeliveries(big\_delivery)
A* (h=MaxAirDist, w=0.500) time: 3.82 \#dev: 3908 total\_cost:
40844.21165 \textbar{}path\textbar{}: 11 path: {[}33919, 18409, 77726,
26690, 31221, 63050, 84034, 60664, 70557, 94941, 31008{]} gas-stations:
{[}31221, 70557{]}

    \subsubsection{Question 17}\label{question-17}

Solve the relaxed deliveries problem. RelaxedDeliveries(big\_delivery)
A* (h=MSTAirDist, w=0.500) time: 1.19 \#dev: 87 total\_cost: 40844.21165
\textbar{}path\textbar{}: 11 path: {[}33919, 18409, 77726, 26690, 31221,
63050, 84034, 60664, 70557, 94941, 31008{]} gas-stations: {[}31221,
70557{]}

    \subsubsection{Question 18}\label{question-18}

    \subsubsection{Question 19}\label{question-19}

\[
\forall x_{i}\in x^{t}:\quad\text{Pr}\left(x_{i}\right)=\frac{\left(\frac{x_{i}}{\alpha}\right)^{-1/T}}{\sum_{j}\left(\frac{x_{j}}{\alpha}\right)^{-1/T}}=\frac{x_{i}^{-1/T}\cdot\alpha^{1/T}}{\sum_{j}\left(x_{j}^{-1/T}\cdot\alpha^{1/T}\right)}=\frac{x_{i}^{-1/T}\cdot\alpha^{1/T}}{\alpha^{1/T}\sum_{j}x_{j}^{-1/T}}=\frac{x_{i}^{-1/T}}{\sum_{j}x_{j}^{-1/T}}
\]

    \subsubsection{Question 20}\label{question-20}

    \subsubsection{Question 21}\label{question-21}

First let us notice that the expression can be rewritten in the form:
\[\forall x_{i}\in x:\quad\text{Pr}\left(x_{i}\right)=\frac{x_{i}^{-1/T}}{\sum_{j\in\left[N\right]}x_{j}^{-1/T}}=\dots=\frac{1}{1+\sum_{i\neq j}\left(\frac{x_{i}}{x_{j}}\right)^{1/T}}\]
And by taking the limit \(T\to0\), we have two options:

\begin{enumerate}
\def\labelenumi{\arabic{enumi}.}
\item
  if \(x_{i}<x_{j}\) for every \(i\neq j\) then
  \(\sum_{i\neq j}\left(\frac{x_{i}}{x_{j}}\right)^{1/T}\xrightarrow[T\to0]{}0\)
  and \(\lim_{T\to0}\text{Pr}\left(x_{i}\right)=\frac{1}{1+0}=1\). This
  is the case where
  \(x_{i}=\min_{j}\left\{ x_{j}\right\} _{j=1}^{N}=\alpha\) is the
  minimal element.
\item
  if there exists at least one \(j\) such that \(x_{i}\geq x_{j}\) then
  we get:
\end{enumerate}

\(\qquad\)(a)
\(\left(\frac{x_{i}}{x_{j}}\right)^{1/T}\xrightarrow[T\to0]{}\infty\) if
\(x_{i}>x_{j}\), which leads to
\(\lim_{T\to0}\text{Pr}\left(x_{i}\right)=\infty\). This is the case
where \(x_{i}\) is not the minimal element of
\(\left\{ x_{j}\right\} _{j=1}^{N}\).

\(\qquad\)(b)
\(\left(\frac{x_{i}}{x_{j}}\right)^{1/T}=\left(1\right)^{1/T}=1\xrightarrow[T\to0]{}1\),
if there is exactly one \(j\) that satisfies the equality
\(x_{i}=x_{j}\).

\(\qquad\)(c) If there are \(\left\{ x_{k}\right\} _{k\in K}\) where
\(K\subseteq N\) and \(2\leq\left|K\right|\leq\left|N\right|\) that
satisfy \(x_{i}=x_{k}\) for all \(k\in K\), and all other coordinates
satisfy \(x_{i}<x_{j}\) (such that their summands vanish), we get
\(\lim_{T\to0}\text{Pr}\left(x_{i}\right)=\frac{1}{1+\left|K\right|\cdot1+\left(0+\dots+0\right)}=\frac{1}{1+K}\).

    \subsubsection{Question 22}\label{question-22}

By using the equivalent expression of the probability function from
question 21 and taking the limit \(T\to\infty\) while assuming
\(x_{j}\neq0\) for all \(j\), we get
\[\lim_{T\to\infty}\text{Pr}\left(x_{i}\right)=\lim_{T\to\infty}\frac{1}{1+\sum_{i\neq j}\left(\underbrace{\frac{x_{i}}{x_{j}}}_{\neq0}\right)^{1/T}}=\frac{1}{1+\underbrace{\left(1+\dots+1\right)}_{N-1\text{ times}}}=\frac{1}{N}\]

and since in our case \(N=5\), we get that indeed the limit is
\(\frac{1}{5}=0.2\), as can be seen in the plot.

    \subsubsection{Question 24}\label{question-24}

    \subsubsection{Question 26}\label{question-26}

    \subsubsection{Question 27}\label{question-27}

Let us consider the following Relaxed Deliveries Heuristic function
\(h\):

For a state \(s\), define the heuristic value \(h(s)\) as the final cost
found by the Relaxed Deliveries Problem with the initial state as the
current state the heuristic function received as input, and the final
state as the current problem's final state. The gas stations, drop
points, current fuel level and full-tank fuel level are all the same as
in the original problem. If the Relaxed Deliveries problem did not find
a solution from the current state to the goal state, let \(h\) return
\(\infty\), or equivalently, some very big number that represents a very
high, undesireable heuristic value, such that the algorithm would rather
not choose.

\(h\) is an admissable heuristic, since its return values always
represent a path whose total cost is comprised of aerial distances,
which are always lower than the true costs the scooter must pay to reach
the goal state. If \(h\) returns \(\infty\), then necessarily there is
no valid path for the strict problem as well, so we might as well treat
its cost as \(\infty\) as well, since we cannot actually get to a goal
state.

    \subsubsection{Question 28}\label{question-28}

The results we got are:

StrictDeliveries(small\_delivery) A* (h=RelaxedProb, w=0.500) time:
16.68 \#dev: 80 total\_cost: 14254.79234 \textbar{}path\textbar{}: 8
path: {[}43516, 67260, 17719, 43454, 43217, 32863, 7873, 42607{]}
gas-stations: {[}17719, 32863{]}

When comparing the result to question 26, we get that for \(w=0.5\), the
total solution cost is the same, but the number of total states expanded
is better by \(33\)\% (120 vs. 80).

for \(w\geq0.58\) we get that the algorithm from question 26 is superior
in terms of states expanded. Taking such a \(w\) will result in a total
cost increase of about 600.

As for the the runtime, the algorithm from question 26 for \(w=0.58\)
took 11.06 seconds to finish. The algorithm in this question tool 16.68
seconds. This is a substantial runtime setback, as it is worse by about
\(50.81\)\% relative to question 26.

    \subsubsection{Theoretical Question}\label{theoretical-question}

\begin{enumerate}
\def\labelenumi{\arabic{enumi}.}
\tightlist
\item
  From the fact that \(h\) is admissible we know that \(\forall s\in S\)
  such that \(\text{Applicable}_h (s)\) is True ,
  \(h_0(h,s)=h(s)\leq h^*(s)\). In addition, \(\forall s\in S\) such
  that \(\text{Applicable}_h (s)\) is False , \(h_0(h,s)=0 \leq h^*(s)\)
  because the price function is bounded from below by \(\delta>0\) so
  \(h^*(s)\geq 0, \forall s\in S\). From that we conclude that \(h_0\)
  is also admissible.
\item
  We assume that the state space is a tree. Our suggested Heuristic is:
\end{enumerate}

\[h'\left(v\right)=\begin{cases}
h\left(v\right) & \text{if Applicable}_{h}\left(v\right)\text{is true}\\
0 & \text{if Applicable}_{h}\left(v\right)\text{is false} \land  \text{isGoal}\left(v\right)\\
\min_{u\in\text{Succ}\left(v\right)}\left(\text{cost}\left(v,u\right)\right) & \text{otherwise}
\end{cases}\text{ is true }\] If \(\text{Applicable}_h (v)\) is True
then \(h'(v) \leq h^*(v)\) because \(h\) is admissible and if
isGoal(\(v\)) is True and \(\text{Applicable}_h(v)\) is False then
\(h'(v) = 0 \leq h^*(v)\) because the function price is positive.
Otherwise, from the fact that the price function is positive we know the
minimal cost of a path from \(v\) to a goal state is at least the
minimal cost of an edge from \(v\) to one of its successors, therefore,
\(h'(v) \leq h^*(v)\) and it means \(h'\) is admissible as needed. Now,
for each \(v\) in the tree \(h_0(v) \leq h'(v)\) because if
\(\text{Applicable}_h (v)\) is True or if isGoal(\(v\)) is True then
\(h_0(v) = h'(v)\) and otherwise \(h_0(v) = 0 < \delta \leq h'(v)\),
\(h'\) is more informed than \(h_0\). 3. As can be seen in section b,
the fact that the state space is a tree was not actually used, therefore
the same heuristic function \(h'\) can be used on any state space.

Another solution could be to add information to each state \(s\) in our
state space; whether it has been visited already or not. If it hasn't
been visited its heuristic value will be \(h'(s)\) (from section b). If
it has been visited then the heuristic value that will be returned is
\(h_0(h,s)\). Since for every state \(s\), \(h'(s)\geq h_0(h,s)\) and
clearly \(h_0(h,s)\geq h_0(h,s)\), in addition to what was defined and
explained in section b, the heuristic function we've suggested is this
section is indeed more informed than \(h_0\), and still admissible.

\begin{enumerate}
\def\labelenumi{\arabic{enumi}.}
\setcounter{enumi}{3}
\tightlist
\item
  The claim is correct.
\end{enumerate}

Let us first note that the \(A^*\) algorithm with the given
\(h_0(h',s)\) heuristic actually behaves like the Uniform Cost
algorithm, since the heuristic value of all nodes is zero (as in Uniform
Cost) except the initial node, and the initial node's heuristic value
doesn't actually affect the algorithm's behaviour since the initial node
does not 'compete' with any other node before being expanded.

From the fact that \(h'(s_0)\) is given (\(s_0\) is the initial state)
and equals \(h^*(s_0)\), we can use a variation of the DFS-L algorithm,
which replaces the depth restriction input with a cost restriction
instead, and use it with a restriction equal to \(h'(s_0)\). the
algorithm in this case is admissible because it is given in the question
that there exists a solution with a cost of \(h'(s_0)\) and it is
optimal, hence the algorithm will necessarily find such a solution, as
we've learned for DFS-L.

    


    % Add a bibliography block to the postdoc
    
    
    
    \end{document}
